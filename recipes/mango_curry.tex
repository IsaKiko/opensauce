%!TEX root = ../sauce_book.tex

% recipe title
\recipe{Mango curry sauce}

% author in small caps
Submitted by: \textsc{Scott Ritchie, University of Melbourne}

% enticement:

\begin{shadequote*}
There are things that are worse than this.
\end{shadequote*}
\hrulefill

% ingredients:
\textbf{Ingredients}

\begin{enumerate}[before=\itshape,font=\normalfont]
\item 1 Large can of mango
\item 1 tblsp. Turmeric powder
\item 1 tsp. Chilli Powder
\item Salt
\item 1 tsp. Sugar
\item Curry Leaves
\item 1 tsp. Mustard seeds
\item Dried red whole chillis
\item 1 tblsp. fenugreek seeds
\item greek yoghurt
\end{enumerate}

\hrulefill

\textbf{Method}

Add oil to a pot and put on high heat. Add one of the mustard seeds to
the pot and cover with lid. Once mustard seed pops, add a pinch to the
pot and recover. Wait until all have popped, then reduce heat to
medium.

Add canned mango to the pot, and bring to boil. Separately, put a small
pan on low heat. Fry dried chillies, curry leaves, chilli powder, and
tumeric until tumeric turns from bright yellow to golden brown.

\textbf{Warning}: if pan is too hot, the chilli powder will start to burn.
This experience is a lot like gassing yourself. Once fried, add spices
to pot of mango, and add salt and sugar to taste. Add fenugreek as
well. Reduce heat to a simmer, and break down mango with wooden spoon.

Once mango is broken down into a thick soup, take off heat and leave to
sit until cooled. Once cooled to room temperature, stir in greek
yoghurt until curry becomes a light yellow color (but not white: this
means you've added too much greek yoghurt). It's important to wait
until the curry has cooled for this last step, otherwise the yoghurt
may curdle. Transfer to fridge, and serve cold with other dishes.

    %\begin{figure}[h!]
    %\centering
    %\addvbuffer[12pt 8pt]{\includegraphics[width=0.65\textwidth]{../images/image.jpg}}
    %\label{fig:recipe}
    %\caption{Template for adding an image}
    %\end{figure}

\vfill
\pagebreak


